% !TeX encoding = UTF-8
% !TeX program = xelatex
% !TeX spellcheck = en_US

\documentclass{cjc}

\usepackage{booktabs}
\usepackage{algorithm}
\usepackage{algorithmic}
\usepackage{siunitx}

\classsetup{
  % 配置里面不要出现空行
  title        = {},
  %title*       = {Title},
  authors      = {
    author1 = {
      name         = {},
      affiliations = {},
      biography    = {},
      % 英文作者介绍内容包括:出生年, 学位(或目前学历), 职称, 主要研究领域(与中文作者介绍中的研究方向一致).
      biography*   = {},
      email        = {},
      phone-number = {},  % 第1作者手机号码(投稿时必须提供,以便紧急联系,发表时会删除)
    },
    author2 = {
      name         = {},
    },
    author3 = {
      name         = {},
      % 通讯作者
      corresponding = true,
    },
  },
  % 论文定稿后,作者署名、单位无特殊情况不能变更。若变更,须提交签章申请,
  % 国家名为中国可以不写,省会城市不写省的名称,其他国家必须写国家名。
  abstract     = {
  },
  abstract*    = {},
  % 中文关键字与英文关键字对应且一致,应有5-7个关键词,不要用英文缩写
  keywords     = {},
  keywords*    = {},
  grants       = {
  },
  % clc           = {TP393},
  % doi           = {10.11897/SP.J.1016.2020.00001},  % 投稿时不提供DOI号
  % received-date = {2019-08-10},  % 收稿日期
  % revised-date  = {2019-10-19},  % 最终修改稿收到日期,投稿时不填写此项
  % publish-date  = {2020-03-16},  % 出版日期
  % page          = 512,
}

\newcommand\dif{\mathop{}\!\mathrm{d}}

% hyperref 总是在导言区的最后加载
\usepackage{hyperref}



\begin{document}

\maketitle
\section{引言}

\section{研究背景}
\section{研究现状}
\subsection{国内外研究现状}
\begin{table}[htbp]
  \centering
  \caption{A}
  \label{tab:exampletable}
  \begin{tabular}{clll}
    \toprule
    \\
    \midrule
  \end{tabular}

\end{table}
\begin{figure}[htbp]
\centering
\includegraphics[width=0.5\textwidth]{figures/5.png}
\caption{}
\label{fig:logo}
\end{figure}
\begin{table}[htbp]
  \centering
  \caption{}
  \label{tab:exampletable}
  \begin{tabular}{clll}
    \toprule
    频数 & 中心度 & 最早出现日期 & 关键词 \\
    \midrule
	107  & 0.35  & 2000  &  polarization\\
	73	 & 0.26	 & 1995	 &  group polarization\\
	57	 & 0.23	 & 2002	 &  attitude\\
	25	 & 0.12	 & 2000	 &  behavior\\
	22	 & 0.05	 & 2008	 &  social identity\\
	21	 & 0.14	 & 2000	 &  conflict\\
	21	 & 0.07	 & 2010	 &  model\\
	19	 & 0.04	 & 2013	 &  prejudice\\
	16	 & 0.04	 & 2000	 &  conformity\\
	15	 & 0.04	 & 2000	 &  perception\\
	14	 & 0.04	 & 2014	 &  ideology\\
	14	 & 0.05	 & 2001	 &  identity\\
	14	 & 0.05	 & 2000	 &  attitude polarization\\
	12	 & 0.03	 & 2001	 &  group decision making\\
	12	 & 0.03	 & 2001	 &  inequality\\
	12	 & 0.07	 & 2000	 &  communication\\
    \bottomrule
  \end{tabular}
\end{table}
\begin{figure}[htbp]
\centering
\includegraphics[width=0.5\textwidth]{}
\caption{利}
\label{fig:logo}
\end{figure}

\section{研究方法}
\nocite{*}

\bibliographystyle{cjc}
\bibliography{example}







\end{document}
